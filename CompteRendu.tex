\documentclass{article}
\usepackage[utf8]{inputenc}

\title{Compte Rendu - Projet Modélisation et Programmation C++}
\author{BAYET Théophile, MOUNE Paul}
\date{May 2017}

\begin{document}

\maketitle

\section{Introduction}

Ce document est le compte-rendu du projet de modélisation et programmation C++. Il regroupe les réponses aux questions des
différents TP ainsi que nos commentaires sur notre implémentation et diverses remarques. \\
L'objectif de ce projet est de modéliser et visualiser un océan.

\vspace{1cm}

\section{TP1}

L'objectif de ce TP est d'implémenter une classe Dvector qui sera ensuite utiliser dans notre modèle pour stocker les différentes hauteurs des points. \\

Nous avons donc créer plusieurs constructeurs ainsi que le destructeur. On retrouve le constructeur par défaut mais aussi un constructeur par recopie et un constructeur par lecture de fichier.

Nous avons aussi implémenter une méthode $display$ qui permet d'afficher sur la sortie standard l'objet de la classe Dvector.

\vspace{0.5cm}
\subsection{Question 1 :}

Considérons le code : \\ \\ $Dvector$ $x;$ \\ $x = Dvector(3, 1.);$ \\ \\

Ici, on construit d'abord un élément de la classe Dvector nommé x sans l'initialiser.
On utilise ensuite l'opérateur = qui a été redéfini pour la classe Dvector et qui va assigner à x une valeur.

\\ \\

En revanche, si l'on écrit directement : \\ \\ $Dvector x = Dvector(3, 1.);$ \\ \\

On construit directement le Dvector x avec une valeur en utilisant le constructeur adéquat que l'on a défini.

\subsection{Question 4 :}

\section{TP2}

Le but de ce TP est d'enrichir notre classe Dvector créee lors du premier TP en implémentant différentes méthodes.

D'une part, il s'agit d'implémenter les opérateurs classiques pour qu'ils fonctionnent sur notre classe, comme par exemple l'addition avec un autre Dvector ou encore la multiplication par un scalaire.

Nous créons aussi un opérateur d'accession à un élément du Dvector. De plus, nous surchargeons les opérateurs d'affectation et de test d'égalité.

\section{TP3}

\section{TP4}

\section{TP5}

\section{Commentaires}


\end{document}
